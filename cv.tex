\documentclass[12pt,a4paper]{moderncv}
\moderncvtheme[blue]{classic}
%\moderncvtheme{banking}
\usepackage[utf8]{inputenc}
\usepackage[scale=0.86]{geometry}
\usepackage{enumerate}

\firstname{Cyrille}
\familyname{Piacibello}
%\title{Développeur}
\AtBeginDocument{\recomputelengths}                     % required when changes are made to page layout lengths

\address{189 Boulevard Brune}{75014 Paris}
\mobile{06 82 21 11 76}
\email{cyrille.piacibello@gmail.com}

\begin{document}
\maketitle

\section{Expériences}
\cventry{Octobre 2015 \\ Aujourd'hui}{Ingénieur
  d'étude}{INRIA}{Bordeaux}{Implémentation d'un solveur
  itératif}{Environnement technique : C, C++, OpenMP, Linux, git}

\cventry{Octobre 2013 \\ Octobre 2015}{Ingénieur
  d'étude}{INRIA}{Bordeaux}{Participation au projet
  ScalFMM}{Environnement technique : C, C++, MPI, OpenMP, Linux, git}

\cventry{Février 2013 \\ Septembre 2013}{Stagiaire}{INRIA}{Bordeaux}{Développement d'un noyau d'approximation de champ lointain basé les séries de Taylor pour une méthode de FMM}{Environnement technique : C, C++, Linux, git}



\section{Diplômes et Études}
\cventry{2013}{Diplôme Ingénieur}{ENSEIRB-MATMECA}{}{Bordeaux}
        {Spécialisation P.R.C.D : Parallélisme, Régulation et Calcul Distribué}.

        \cventry{2010}{Classes Préparatoires}{Lycée Massena}{}{Nice}{}

        \section{Compétences}
        \subsection{Informatique}
        \cvcomputer{\underline{Langages :}}{C, C++ (norme 2011)}{\underline{Standards :}}{BLAS, LAPACK, MPI, OpenMP}
        \cvcomputer{\underline{Outils :}}{Shell, \LaTeX, Git, Svn}{}{}

        \subsection{Langues}
        \cvlanguage{Anglais}{lu, écrit, parlé}{Rédaction documentaire. TOEIC 800 en 2012}

        \renewcommand{\listitemsymbol}{-} % change the symbol for lists

        \section{Détails des expériences}
        \cventry{INRIA}{Ingénieur d'étude}{IB-BGMRes-DR : Implémentation d'un solveur itératif}{}{}{
          \begin{enumerate}[$\bullet$]
            \setlength\itemsep{0px}
          \item Implémentation d'un GMRes (multi second membre) par bloc avec
            produit matrice vecteur externe.
          \item Détection et prise en compte d'Inexact Breakdown (convergence
            partielle parmi les seconds membres).
          \item Deflation au restart (recyclage d'information au restart).
          \end{enumerate}
          Environnement technique: C++, C, Git\\
          %\smallskip
        }
        \cventry{INRIA}{Ingénieur d'étude}{Participation au projet ScalFMM, bibliothèque générique implémentant l'algorithme de Fast Multipole Method}{}{}{
          \begin{enumerate}[$\bullet$]
            \setlength\itemsep{0px}
          \item Mise en place d'un ordonnancement par tâche (Standard OpenMP 4.0)
          \end{enumerate}
          Environnement technique: C++, C, MPI, OpenMP, Git\\
          \smallskip
        }


        % Publications from a BibTeX file
        \nocite{*}
        \bibliographystyle{plain}
        \bibliography{publications}       % 'publications' is the name of a BibTeX file

\end{document}
