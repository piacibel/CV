\documentclass[12pt,a4paper]{letter}
\usepackage[utf8]{inputenc}

\signature{Cyrille Piacibello}
\address{36 rue Carnot,\\ 33400 Talence}

\begin{document}
\begin{letter}{Inria Bordeaux Sud-Ouest,\\
    200 avenue de la vieille tour, \\
    33400 Talence}

  \opening{Objet : Lettre de Motivation pour le poste d'Ingénieur sur
    le projet Hi-BOX.}

  Bonjour,

  Je participe depuis deux ans maintenant au projet Hi-BOX,
  rassemblant des partenaires publics et privés dans le but de
  produire un solveur générique pour des problèmes d'acoustique et
  d'électro-magnétisme. Dans ce cadre là, j'ai travaillé en tant
  qu'ingénieur tout d'abord à l'industrialisation d'un logiciel de
  l'équipe HiePACS (ScalFMM), pour qu'il soit utilisable dans le cadre
  du projet, et j'implémente maintenant une méthode itérative de
  résolution de système avec plusieurs seconds membres.

  Ces projets m'ont permis à la fois de travailler dans un cadre de
  recherche et de performance très pointu, et également de m'initier
  aux problématiques industrielles.

  Continuer à travailler dans ce cadre est pour moi une opportunité
  extraordinaire. Ce poste me permet de me former scientifiquement
  auprès de spécialistes reconnus et d'appréhender le monde de la
  simulation industrielle.

  Ma formation d'ingénieur est en adéquation avec la mission, d'autant
  plus que ces 6 mois seront consacrés à l'optimisation et à la
  parallélisation du solveur itératif, domaine dans lequel je me suis
  spécialisé au cours de mes études.

  Je pense que mon expérience dans le domaine et ma formation me
  permettent donc de prétendre à ce poste.

  \closing{Cordialement}


\end{letter}
\end{document}
