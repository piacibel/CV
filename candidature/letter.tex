%%%%%%%%%%%%%%%%%%%%%%%%%%%%%%%%%%%%%%%%%
% Long Lined Cover Letter
% LaTeX Template
% Version 2.0 (September 17, 2021)
%
% This template originates from:
% https://www.LaTeXTemplates.com
%
% Authors: Wlving Lee
% (lee.wlving@connect.um.edu.mo)
%
% License:
% CC BY-NC-SA 4.0 (https://creativecommons.org/licenses/by-nc-sa/4.0/)
%
%%%%%%%%%%%%%%%%%%%%%%%%%%%%%%%%%%%%%%%%%

%----------------------------------------------------------------------------------------
%	PACKAGES AND OTHER DOCUMENT CONFIGURATIONS
%----------------------------------------------------------------------------------------

\documentclass{article}

\usepackage[utf8]{inputenc}
\usepackage[french]{babel}
\usepackage[T1]{fontenc}

\usepackage{charter} % Use the Charter font

\usepackage[
	a4paper, % Paper size
	top=1in, % Top margin
	bottom=1in, % Bottom margin
	left=1in, % Left margin
	right=1in, % Right margin
	%showframe % Uncomment to show frames around the margins for debugging purposes
]{geometry}

\setlength{\parindent}{0pt} % Paragraph indentation
\setlength{\parskip}{1em} % Vertical space between paragraphs

\usepackage{graphicx} % Required for including images

\usepackage{fancyhdr} % Required for customizing headers and footers

\fancypagestyle{firstpage}{%
	\fancyhf{} % Clear default headers/footers
	\renewcommand{\headrulewidth}{0pt} % No header rule
	\renewcommand{\footrulewidth}{1pt} % Footer rule thickness
}

\fancypagestyle{subsequentpages}{%
	\fancyhf{} % Clear default headers/footers
	\renewcommand{\headrulewidth}{1pt} % Header rule thickness
	\renewcommand{\footrulewidth}{1pt} % Footer rule thickness
}

\AtBeginDocument{\thispagestyle{firstpage}} % Use the first page headers/footers style on the first page
\pagestyle{subsequentpages} % Use the subsequent pages headers/footers style on subsequent pages

%----------------------------------------------------------------------------------------

\begin{document}

%----------------------------------------------------------------------------------------
%	FIRST PAGE HEADER
%----------------------------------------------------------------------------------------

% \includegraphics[width=0.4\textwidth]{UMiPtEng_B_WH_RGB_01.png} % Logo

\vspace{-1em} % Pull the rule closer to the logo

\rule{\linewidth}{1pt} % Horizontal rule

\bigskip\bigskip % Vertical whitespace

%----------------------------------------------------------------------------------------
%	YOUR NAME AND CONTACT INFORMATION
%----------------------------------------------------------------------------------------

\hfill
\begin{tabular}{l @{}}
	\today \bigskip\\ % Date
	PIACIBELLO Cyrille \\
	Appt 114, Entrée D,  234 rue de Suzon \\ % Address
	33400 Talence \\
	06-82-21-11-76 \\
	cyrille.piacibello@gmail.com
\end{tabular}

\bigskip % Vertical whitespace

%----------------------------------------------------------------------------------------
%	ADDRESSEE AND GREETING
%----------------------------------------------------------------------------------------

\begin{tabular}{@{} l}
  Service Recrutement \\
  ARTAL Technologie \\
  1 rue Ariane,\\
  31500, Ramonville-Saint-Agne
\end{tabular}

\bigskip % Vertical whitespace

OBJET : Lettre de motivation pour un poste de développeur C++ Spatial.


\bigskip % Vertical whitespace

%----------------------------------------------------------------------------------------
%	LETTER CONTENT
%----------------------------------------------------------------------------------------

Bonjour,

Je souhaite saisir l'opportunité de travailler chez ARTAL Technologies
dans la branche spatiale. Je pense que mon expérience dans le
développement logiciel dans le contexte spatial au sein du Segment Sol
de la mission Euclid ainsi que mon expérience du développement en C++
dans un environnement de recherche peuvent être valorisées au sein de
votre entreprise.

Dans le cadre de mon dernier emploi au CNRS, j'ai travaillé dans une
équipe internationale dans un environnement qui s'est graduellement
transformé avec l'ajout de tests, d'intégration continue, puis de
déploiement automatique notament sur des clusters de calcul. J'ai
accompagné l'évolution de l'environnement en parallèle des cycles de
développement. Le simulateur auquel je contribuais était écrit
intégralement en python, et notre travail était régulièrement suivi en
terme de qualité et de performance par les équipes de soutien du CNES.

Précédemment, à l'INRIA, j'ai travaillé dans des équipes plus
restreintes, avec plus de d'autonomie. J'ai contribué à deux
bibliothèques C++ de calcul numérique au sein d'une ANR pilotée par la
DGA en partenariat avec des acteurs industriels. Dans ce projet, la
performance des briques logicielles était primordiale.



\bigskip % Vertical whitespace

Je vous remercie de l'attention que vous porterez à ma candidature,

\vspace{50pt} % Vertical whitespace

Cyrille PIACIBELLO

\end{document}
