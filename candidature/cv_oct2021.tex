\documentclass[12pt,a4paper]{moderncv}
\moderncvtheme[blue]{classic}
%\moderncvtheme{banking}
\usepackage[utf8]{inputenc}
\usepackage[scale=0.86]{geometry}
\usepackage{enumerate}

\firstname{Cyrille}
\familyname{Piacibello}
\title{Développeur}
\AtBeginDocument{\recomputelengths}                     % required when changes are made to page layout lengths

\address{Résidence le Voltaire}{Appt 114, Entrée D}{234 rue de Suzon, 33400 Talence}
\mobile{06 82 21 11 76}
\email{cyrille.piacibello@gmail.com}

\begin{document}
\maketitle

\section{Expériences}
\cventry{Avril 2017 \\ Octobre 2021}{Ingénieur
  d'étude}{IAP-CNRS}{Paris}{Participation au développement du
  simulateur de l'instrument VIS de la mission spatial
  EUCLID}{Environnement technique : Python, Linux, git, Jenkins,
  Redmine, SonarQube}

\cventry{Octobre 2015 \\ Mars 2017}{Ingénieur
  d'étude}{INRIA}{Bordeaux}{Implémentation d'un solveur
  itératif}{Environnement technique : C, C++, OpenMP, Linux, Git}

\cventry{Octobre 2013 \\ Octobre 2015}{Ingénieur
  d'étude}{INRIA}{Bordeaux}{Participation au projet
  ScalFMM}{Environnement technique : C, C++, MPI, OpenMP, Linux, Git}

\section{Compétences}
\subsection{Informatique}
\cvcomputer{\underline{Langages :}}{C, C++ (norme 2011), Python}{\underline{Standards :}}{BLAS, LAPACK, MPI, OpenMP}
\cvcomputer{\underline{Outils :}}{Shell, \LaTeX, Git, Valgrind}{}{}

\subsection{Langues}
\cvlanguage{Anglais}{lu, écrit, parlé}{Rédaction documentaire. TOEIC 800 en 2012}

\renewcommand{\listitemsymbol}{-} % change the symbol for lists


\section{Diplômes et Études}
\cventry{2013}{Diplôme Ingénieur}{ENSEIRB-MATMECA}{}{Bordeaux}
        {Spécialisation P.R.C.D : Parallélisme, Régulation et Calcul Distribué}.
\cventry{2010}{Classes Préparatoires}{Lycée Massena}{}{Nice}{}


\newpage

\section{Détails des expériences}
\cventry{IAP-CNRS}{Ingénieur d'étude}{Participation au développement du
  simulateur de l'instrument VIS de la mission spatial
  EUCLID}{}{}{
  \begin{enumerate}[$\bullet$]
    \setlength\itemsep{0px}
  \item Simulation d'effets instrumentaux à partir de modèle et à
    partir de mesures expérimentales réalisées sur le détecteur VIS.
  \item Validation des effets implémentés.
  \item Contribution au Challenge Scientifique. (Passage à l'echelle
    du wrapper)
  \end{enumerate}
  Environnement technique: Python, Git, Jenkins, Redmine, Slurm\\
  % \smallskip
}

\cventry{INRIA}{Ingénieur d'étude}{IB-BGMRes-DR : Implémentation d'un solveur itératif}{}{}{
  \begin{enumerate}[$\bullet$]
    \setlength\itemsep{0px}
  \item Implémentation d'un solveur GMRes (multi second membre) par bloc avec
    produit matrice vecteur externe.
  \item Détection et prise en compte d'Inexact Breakdown (convergence
    partielle parmi les seconds membres).
  \item Deflation au restart (recyclage d'information au restart).
  \end{enumerate}
  Environnement technique: C++, C, Git\\
  % \smallskip
}
\cventry{INRIA}{Ingénieur d'étude}{Participation au projet ScalFMM, bibliothèque générique implémentant l'algorithme de Fast Multipole Method}{}{}{
  \begin{enumerate}[$\bullet$]
    \setlength\itemsep{0px}
  \item Mise en place d'un ordonnancement par tâche (Standard OpenMP 4.0)
  \end{enumerate}
  Environnement technique: C++, C, MPI, OpenMP, Git\\
  \smallskip
}


% Publications from a BibTeX file
\nocite{*}
\bibliographystyle{plain}
\bibliography{publications}       % 'publications' is the name of a BibTeX file

\end{document}
