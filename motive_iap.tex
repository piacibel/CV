\documentclass[12pt,a4paper]{letter}
\usepackage[utf8]{inputenc}


\signature{Cyrille Piacibello}
\address{189 Boulevard Brune\\ 75014 Paris}
\telephone{06 82 21 11 76}

\begin{document}
\begin{letter}{Institut d'Astrophysique de Paris,\\
    98 bis boulevard Arago, \\
    75014 Paris}

  \vspace*{-10\baselineskip}

  \opening{Objet : Lettre de Motivation pour le poste d'ingénieur en
    Développement Logiciel dans le cadre de la mission Euclid.}

  Bonjour,

  Je suis intéressé par l'offre de développeur au sein du projet
  Euclid qui a été diffusée sur la liste calcul. Je trouve la
  perspective de travailler dans le domaine de l'astrophysique
  particulièrement stimulante.

  Le traitement d'image n'est pas ma spécialité. Néanmoins, je pense
  que mes expériences dans le développement d'applications
  scientifiques peuvent se révéler utiles :

  Dans le logiciel ScalFMM, logiciel implémentant la Fast Multipole
  Method, je me suis principalement confronté aux problématiques
  d'équilibrage de travail (en mémoire partagée et distribuée), ainsi
  qu'à l'industrialisation de la bibliothèque, dans le cadre d'un
  projet de boite à outils pour traiter des problèmes industriels
  d'acoustique et d'électromagnétisme.

  Récemment, le développement de la bibliothèque IB-BGMRes-DR m'a
  quant à lui apporté deux choses :
  \begin{itemize}
  \item L'expérience de la construction d'une bibliothèque C++
    d'algèbre linéaire depuis zéro, dans laquelle j'ai fait seul les
    choix d'architecture.
  \item Des connaissances approfondies sur les méthodes itératives de
    résolution de système, notamment grâce à la collaboration avec les
    scientifiques qui ont conçu la méthode que j'ai implémenté.
  \end{itemize}


  Je pense que mon profil peut vous intéresser et me tiens à votre
  disposition pour un éventuel entretien.

  \closing{Cordialement,}


\end{letter}
\end{document}
