\documentclass[12pt,a4paper]{moderncv}
\moderncvtheme[blue]{classic}
\usepackage[utf8]{inputenc}
\usepackage[scale=0.86]{geometry}
\usepackage{enumerate}

\firstname{Matias}
\familyname{Hastaran}
%\title{Développeur}
\AtBeginDocument{\recomputelengths}                     % required when changes are made to page layout lengths

\address{12 rue d'alembert}{33000 Bordeaux}
\mobile{06 48 05 41 45}
\email{hastaran.matias@gmail.com}

\begin{document}
\maketitle

\section{Expériences}
\cventry{Octobre 2013 \\ Aujourd'hui}{Ingénieur d'étude}{INRIA}{Bordeaux}{Participation au projet EZTrace}{Environnement technique : C, ASM, MPI, OpenMP, Linux, git}
\cventry{Janvier 2013 \\ Octobre 2013}{Ingénieur d'étude}{ISEO}{Bidart}{Développement de nouveaux modules sur le logiciel existant}{Environnement technique : C, C++, Oracle, Linux, Windows, Visual SourceSafe}
\cventry{Janvier 2012 \\ Août 2012}{Stagiaire R\&D}{OpenWide Ingenierie}{Paris}{Participation au projet de R\&D PARSEC}{Environnement technique : C, Assembleur x86, Linux}
\cventry{Août 2011 \\ Décembre 2011}{Développeur C++}{APSIDE : Mission chez Generals Electrics}{Buc}{Mise à jour d'une application de mammographie pour support de la 3D}{Environnement technique : C++, ClearCase, Solaris}
\cventry{Octobre 2010 \\ Août 2011}{Développeur R\&D}{Ensuite-Info}{Chilly-Mazarin}{Développement de démonstrateurs pour des processeurs d'analyse situationnelle}{Environnement technique : Linux, C++, Git}
\cventry{Janvier 2009 \\ Avril 2009}{Administrateur réseau}{Service Départemental d’Incendie et de Secours des Pyrénées-Atlantiques}{Pau}{Audit du réseau. Mise en place de solutions de sauvegarde}{Environnement technique : Linux, Cisco, Bash}
\cventry{Mai 2008 \\  Août 2008}{Stagiaire}{Service Départemental d’Incendie et de Secours des Pyrénées-Atlantiques}{Pau}{Mise en place de solutions de supervision : NAGIOS, CACTI}{Environnement technique : Linux, Cisco, PHP, C, Bash}
%\cventry{Octobre 2007 \\ Janvier 2008}{Stagiaire}{Agence Départementale du Numérique 64}{Pau}{Développement d'une application de gestion de frais}{Environnement technique : Linux, PHP, MySQL, HTML}

\section{Diplômes et Études}
\cventry{2012}{Master informatique}{EPITECH}{}{}{}
\cventry{2008}{Diplôme d'analyste programmeur}{EXIA}{}{}{}

\section{Compétences}
\subsection{Informatique}
%\cvcomputer{\underline{Conception :}}{Merise, UML}{\underline{Base de données :}}{Oracle, MySQL}
\cvcomputer{\underline{Langages :}}{C, C++, Python, Assembleur x86}{\underline{Librairies :}}{Qt, Flex, Bison}
\cvcomputer{\underline{Systèmes :}}{ArchLinux, Debian, FreeBSD, Solaris, POK}{\underline{Réseaux :}}{Certification CISCO CCNA1 et CCNA 2}
\cvcomputer{\underline{Outils :}}{Git, SVN, ClearCase, DOORS, Qemu, Latex}{\underline{Standards :}}{MPI, OpenMP, ARINC653}
\subsection{Langues}
\cvlanguage{Anglais}{lu, écrit}{Rédaction documentaire. TOEIC 800 en 2010}\\

\renewcommand{\listitemsymbol}{-} % change the symbol for lists
\section{Détails des expériences}

\cventry{INRIA}{Ingénieur d'étude}{Pariticipation au projet EZTrace, logiciel permettant la génération de trace d'exécution.}{}{}{
  \begin{enumerate}[$\bullet$]
    \setlength\itemsep{0px}
  \item Stabilisation de l'existant (chaine de compilation, module MPI...)
  \item Ajout d'un module pour la gestion du framework StarPU
  \item Création de nouveaux outils de statistiques
  \item Ajout d'une suite de tests et d'une plateforme d'intégration continue
  \item Participation à des formations EZTrace
  \end{enumerate}
  Environnement technique: C, ASM x86, MPI, OpenMP, Git, autoconf, cmake, GCC, Linux\\
  \smallskip
}

\cventry{ISEO}{Ingénieur d'étude}{Développement de nouveaux modules sur l'application d'ISEO. Cette application, qui a pour but la surveillance de la qualité de l'air, récupère, traite, et affiche des données en provenance d’analyseurs.}{}{}{
  \begin{enumerate}[$\bullet$]
    \setlength\itemsep{0px}
  \item Création d'un module permettant la génération de fichiers de configuration de capteurs et le traitement de données renvoyées par ceux-ci via un serveur FTP.
  \item Création d'un module permettant la récupération et la reconstruction de données d'un analyseur en cas de coupure de communication.
  \item Création d'outils pour la gestion et la diffusion de mise à jour sur les sites clients.
  \item Débogage des modules dédiés aux calculs, à la génération de rapports et au transfert des fichiers de configuration.
  \end{enumerate}
  Environnement technique: C, C++, libcurl, Oracle, Visual Studio, Visual SourceSafe, GCC, Linux, Windows\\
  \smallskip
}

 \cventry{OpenWide Ingénierie}{R\&D}{Participation au projet de R\&D PARSEC qui à pour but la création d'outils de développement pour des systèmes critiques temps-réel distribués respectant des standard comme ARINC653}{}{}{
   \begin{enumerate}[$\bullet$]
     \setlength\itemsep{0px}
   \item  Développement kernel
   \item  Modification du kernel POK
     \begin{enumerate}[$\bullet$]
       \setlength\itemsep{0px}
     \item Etude du kernel
     \item Ajout de l’API de la norme ARINC653
     \item Assurer les contraintes temps réel
     \end{enumerate}
   \item  Création d'un parseur entre le logiciel Syndex vers le langage C
     \begin{enumerate}[$\bullet$]
       \setlength\itemsep{0px}
     \item Création des analyseurs lexical et syntaxique avec flex et Bison
     \item Implémentation du parseur proprement dit en C++ via le pattern Visitor
     \end{enumerate}
   \item  Environnement technique : C, C++, Assembleur x86, GIT, svn, CMake, Linux, POK
 \end{enumerate}
 \smallskip
 }

 \cventry{General Electrics Healthcare}{Développeur C++}{Participation à l'évolution d'un logiciel de mammographie}{}{}{
   \begin{enumerate}[$\bullet$]
     \setlength\itemsep{0px}
   \item  Ajout de nouvelles fonctionnalités diverses (contrôles de dosages de rayons, modification d’interface, communication entre divers modules)
     \begin{enumerate}[$\bullet$]
       \setlength\itemsep{0px}
     \item Implémentation du code (beaucoup de retro-engineering)
     \item Modification d’interfaces en TK/TCL
     \item Rédaction des tests plans
     \end{enumerate}
   \item Stabilisation de l’existant
   \item Transmission de compétences aux nouveaux arrivants
   \item Modification de la chaine de compilation via la modification de script CSH
   \item Environnement technique : C++, CSH, ClearCase, DOORS, svn, Solaris
   \end{enumerate}
   \smallskip
 }

 \cventry{Ensuite-Info}{R\&D}{Entreprise spécialisée en analyse situationnelle. Développement d'outils autour de processeurs logiciel d'analyse situationnelle afin de pouvoir faire des démonstrations aux clients}{}{}{
   \begin{enumerate}[$\bullet$]
     \setlength\itemsep{0px}
   \item Définition des besoins afin de pouvoir réaliser un démonstrateur pertinant.
   \item Développement des démonstrateurs
     \begin{enumerate}[$\bullet$]
       \setlength\itemsep{0px}
     \item Alectryon : Outil de veille technologique ; le processeur qualifie des URLs en fonction d’un domaine de recherche défini par l’utilisateur et une application tierce présente les résultats triés selon un indice de pertinence attribué par le processeur.
       \item Duquenne : Analyse de diagnostics médicaux ; le processeur est capable de détecter les singularités (mettre en évidence les traitements les mieux adaptés à une pathologie, détecter des comportements anormaux...)
     \end{enumerate}
   \item Réalisation de tests unitaires pour le processeur
   \item Environnement technique : C++, Python, Qt, CMake, Git, Linux
   \end{enumerate}
\bigskip
}

% Publications from a BibTeX file
\nocite{*}
\bibliographystyle{plain}
\bibliography{publications}       % 'publications' is the name of a BibTeX file

\end{document}
